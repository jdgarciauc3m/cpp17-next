\subsection{Otras modificaciones sobre atributos}

\begin{frame}[t,fragile]{Atributos en nuevos lugares}
\begin{itemize}
  \item Se añaden atributos a nuevos elementos:
    \begin{itemize}
      \item Espacios de nombre:
\begin{lstlisting}
namespace [[deprecated]] mi_lib_anterior {
  void f1();
  //...
}

void g() {
  mi_lib_anterior::f1(); // Warning: deprecated
  //...
}
\end{lstlisting}

      \item Enumeradores de un enumerado:
\begin{lstlisting}
enum modo {
  apagado,
  enciendiendo [[deprecated]],
  encendido
};	
\end{lstlisting}
    \end{itemize}
\end{itemize}
\end{frame}

\begin{frame}[t,fragile]{Atributos desconocidos ignorados}
\begin{itemize}
  \item Antes de C++17 un compilador podía dar un error para un atributo
        que no reconocía (p.ej. de otro compilador):
\begin{lstlisting}
void f(void * p) {
  [[gsl::supress("type")]]  // Solamente reconocido por clang-tidy
  int * q = reinterpret_cast<int*>(p);
  //...
}
\end{lstlisting}
  \vfill
  \item Ahora no puede dar un error.
    \begin{itemize}
      \item Aunque si un warning.
    \end{itemize}
\end{itemize}
\end{frame}

\begin{frame}[t,fragile]{Espacios de nombre y atributos}
\begin{itemize}
  \item En C++11/14 los espacios de nombre de atributos deben cualificar
        explícitamente cada atributo.
\begin{lstlisting}
void f(X x, Y y) {
  [[rpr::kernel, rpr::target(cpu,gpu,fpga), rpr::in(x), rpr::out(y)]]
  do_f(x,y);
  //...
}
\end{lstlisting}

  \vfill\pause
  \item En C++17 se puede evitar la repetición:
\begin{lstlisting}
void f(X x, Y y) {
  [[using rpr: kernel, target(cpu,gpu,fpga), in(x), out(y)]]
  do_f(x,y);
  //...
}
\end{lstlisting}

\end{itemize}
\end{frame}
