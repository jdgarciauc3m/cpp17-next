\subsection{Orden de evaluación}

\begin{frame}[t,fragile]{Orden de evaluación}
\begin{itemize}
  \item Orden de evaluación de operandos en C++11:
\begin{lstlisting}[basicstyle=\scriptsize]
void f() {
  std::string s = "but I have heard it works even if you don't believe in it";
  s.replace(0, 4, "").replace(s.find("even"), 4, "only").replace(
      s.find(" don't"), 6, "");
  assert(s == "I have heard it works only if you believe in it");
  //...
}
\end{lstlisting}

  \vfill\pause
  \item \textbad{Problema}:
    \begin{itemize}
      \item Encadenamiento de llamadas a funciones miembro.
      \item Orden ejecución no definido en C++11.
      \item Error en \emph{``The C++ Programming Language. 4 Ed.''}.
    \end{itemize}
\end{itemize}
\end{frame}

\begin{frame}[t,fragile]{Orden de evaluación}
\begin{itemize}
  \item Otro ejemplo en C++11:
\begin{lstlisting}
void f() {
  std::map<int,int> v;
  v[0] = v.size();
  //...
}
\end{lstlisting}
  \item \textmark{Resultado}:
    \begin{itemize}
      \item \cppid{\{0,0\}}.
      \item \cppid{\{0,1\}}.
    \end{itemize}
\end{itemize}
\end{frame}

\begin{frame}[t,fragile]{Fijando el orden en C++17}
\begin{itemize}
  \item Se fija el orden de evaluación en:
    \begin{itemize}
      \item \cppid{a.b}
      \item \cppid{a->b}
      \item \cppid{a->*b}
      \item \cppid{a(b1,b2,b3)}
      \item \cppid{b op= a}
      \item \cppid{a[b]}
      \item \cppid{a <{}< b}
      \item \cppid{a >{}> b}
    \end{itemize}

  \vfill\pause
  \item Pero:
    \begin{itemize}
      \item \textbad{No se define el orden de evaluación de argumentos de función}.
    \end{itemize}
\end{itemize}
\end{frame}
