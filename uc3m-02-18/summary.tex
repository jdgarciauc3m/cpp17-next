\section{Resumen}

\begin{frame}[t]{Resumen}
\begin{itemize}[<+->]
  \item Limpieza de características antiguas.
  \item C++ es un lenguaje en constante evolución.
  \item Un lenguaje algo más predecible.
  \item Mejor soporte (ligeramente) de programación genérica.
  \item Mayor uso de atributos.
  \item Simplificaciones en el lenguaje para código más claro.
  \item Algoritmos paralelos y acceso al sistema de ficheros.
  \item Algunos tipos nuevos en la biblioteca.
\end{itemize}
\end{frame}

\begin{frame}[t]{¿Y después?}
\begin{itemize}
  \item Ya estamos trabajando en C++20:
    \begin{itemize}
      \item Mejor programación genérica con \textgood{conceptos} {B. Stroustrup et al.}
      \item Mejor procesamiento asíncrono con \textgood{corrutinas} {G. Nishanov}
      \item Mejor estructuración con \textgood{módulos} {G. Dos Reis}
      \item Mejor detección de defectos con \textgood{contratos} {J. Daniel García et al.}
      \item \ldots
    \end{itemize}
\end{itemize}
\end{frame}

\begin{frame}[t]{La fundación C++: \textbf{\url{http://isocpp.org}}}
\includegraphics[width=\textwidth]{../img/isocpp.png}
\end{frame}

\begin{frame}[t]{Otras posibles charlas: C++11/14}
\begin{itemize}
  \item El sistema de tipos.
  \item Iniciación uniforme.
  \item Expresiones constantes y tipos literales.
  \item Simplificando mis clases.
  \item Expresiones lambda.
  \item Plantillas y número variable de argumentos.
  \item Introducción a la meta-programación.
  \item Smart pointers.
  \item Introducción a la concurrencia.
  \item Uso avanzado de hilos y mutex.
  \item Introducción a la programación libre de cerrojos.
  \item Futuros y promesas.
  \item \ldots
\end{itemize}
\end{frame}
