\subsection{Elusión de copia}

\begin{frame}[t]{Eludiendo copias temporales}
\begin{itemize}
  \item El estándar \textmark{permite evitar} copias intermedias en algunso casos:
    \begin{itemize}
      \item Cuando se inicia un objeto con otro objeto temporal.
      \item Cuando se devuelve una variable local.
      \item Cuando se captura una excepción por valor.
    \end{itemize}
  \vfill\pause
  \item Pero:
    \begin{itemize}
      \item Es una optimización opcional.
      \item Hace falta definir constructores de copia por si el compilador no optimiza.
    \end{itemize}
\end{itemize}
\end{frame}

\begin{frame}[t,fragile]{Ejemplo de elusión}
\begin{lstlisting}
class X {
public:
  X(int n);
  X(const X &) = delete;
  X(X &&) = delete;
  //...
};

X make_x(int z) { return X{z}; }

void f() {
  auto a = make_x(10); // Sin copias
}
\end{lstlisting}
\end{frame}
