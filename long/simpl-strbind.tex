\subsection{Vinculación estructurada}

\begin{frame}[t,fragile]{Múltiples valores de retorno}
\begin{itemize}
  \item En C++14, se pueden gestionar múltiples valores de retorno con
        \cppid{std::tuple} y \cppid{std::tie}.
\begin{lstlisting}
std::tuple<int, string, bool> f() {
  return make_tuple(1, "hola", false);
}

void g() {
  int n;
  std::string msg;
  bool result;
 
  std::tie(n, msg, result) = f();
  //...
}
\end{lstlisting}
\end{itemize}
\end{frame}

\begin{frame}[t,fragile]{Vinculación estructurada}
\begin{itemize}
  \item En C++17, se introducen las \textmark{vinculaciones estructuradas}:
\begin{lstlisting}
std::tuple<int, string, bool> f() {
  return make_tuple(1, "hola", false);
}

void g() {
  auto [n, msg, result] = f();
  //...
}
\end{lstlisting}
\end{itemize}
\end{frame}

\begin{frame}[t,fragile]{Otras vinculaciones estructuradas}
\begin{itemize}
  \item Las vinculaciones estructuradas no están limitadas a tuplas.
    \begin{itemize}
      \item Arrays:
\begin{lstlisting}
double v[] = {1, 2, 3};
auto [a,b,c] = v;
\end{lstlisting}

      \vfill\pause
      \item Para cualquier tipo que soporte \cppid{tuple\_size<>} y \cppid{get<N>()}
            (p. ej. \cppid{std::pair}).
\begin{lstlisting}
pair<int,string> f();

auto [x,y] = f();
\end{lstlisting}

      \vfill\pause
      \item Estructuras sin datos miembro estáticos.
\begin{lstlisting}
struct X { int a; double b; char c; };
X f();
const auto [x,y,z] = f();
\end{lstlisting}
    \end{itemize}
\end{itemize}
\end{frame}

\begin{frame}[t,fragile]{Combinando vinculaciones estructuradas y bucles de rango}
\begin{itemize}
  \item Se pueden combinar las vinculaciones estructuradas con el uso
        de bucles for basados en rango.
\begin{lstlisting}
std::map<int, std::string> dic;
//...
for (auto & [id, nombre] : dic) {
  cout << "Id: " << id << "\n";
  cout << "Nombre: " << nombre << "\n";
}
\end{lstlisting}
\end{itemize}
\end{frame}
