\subsection{Algoritmos no numéricos}

\begin{frame}[t]{Algoritmos no numéricos}
\begin{itemize}
  \item Son algoritmos que no requieren que los valores accedidos por
        los iteradores sean de un tipo numérico.
    \begin{itemize}
      \item Más generales que los algoritmos numéricos.
    \end{itemize}
  \item Ejemplos:
    \begin{itemize}
      \item \cppid{for\_each()}.
      \item \cppid{for\_each\_n()}.
    \end{itemize}
\end{itemize}
\end{frame}

\begin{frame}[t,fragile]{For Each}
\begin{lstlisting}g
template <class ExecutionPolicy, class InputIterator, class Function>
void for_each(ExecutionPolicy && ep, 
              InputIterator first, InputIterator last,
              Function f);
\end{lstlisting}
\begin{itemize}
  \item Aplica \cppid{f} al resultado de desreferenciar cada iterador en el rango \cppid{[first,last)}.
\end{itemize}
\begin{block}{Ejemplo}
\begin{lstlisting}g
long cuenta_primos(const vector<int> & v) {
  std::atomic<long> count;
  for_each(par, begin(v), end(v), [&](int x) {
    if (esprimo(x)) { count++; }
  });
  return count;
}
\end{lstlisting}
\end{block}
\end{frame}

\begin{frame}[t,fragile]{For Each}
\begin{lstlisting}g
template <class ExecutionPolicy, class InputIterator, class Size, class Function>
void for_each_n(ExecutionPolicy && ep, 
              InputIterator first, Size n,
              Function f);
\end{lstlisting}
\begin{itemize}
  \item Aplica \cppid{f} al resultado de desreferenciar cada iterador en el rango \cppid{[first, first+n)}.
\end{itemize}
\begin{block}{Ejemplo}
\begin{lstlisting}g
long cuenta_primos(const vector<int> & v, int k) {
  std::atomic<long> count;
  for_each_(par, begin(v), k, [&](int x) {
    if (esprimo(x)) { count++; }
  });
  return count;
}
\end{lstlisting}
\end{block}
\end{frame}
