\subsection{Algortimos numéricos}

\begin{frame}[t]{Algoritmos numéricos}
\begin{itemize}
  \item Algoritmos orientados a secuencias numéricas.
    \begin{itemize}
      \item \cppid{reduce}.
      \item \cppid{exclusive\_scan}.
      \item \cppid{inclusive\_scan}.
      \item \cppid{transform\_reduce}.
      \item \cppid{transform\_exclusive\_scan}.
      \item \cppid{transform\_inclusive\_scan}.
    \end{itemize}
  \vfill
  \item También se añaden las versiones sin parámetro de política.
\end{itemize}
\end{frame}

\begin{frame}[t]{Operaciones básicas}
\begin{itemize}
  \item Operaciones usadas en la definición de la semántica de algoritmos.
  \item \cppid{GENERALIZED\_SUM} y \cppid{GENERALIZED\_NONCOMMUTATIVE\_SUM}.
    \begin{itemize}
      \item Conceptualmente suma (aunque aplicable a cualquier operador).
    \end{itemize}
  \item \cppid{GENERALIZED\_SUM(op, a1, \ldots, aN)}:
    \begin{itemize}
      \item Concepto de suma commutativa.
      \item Puede aplicar la operación en cualquier orden.
    \end{itemize}
    \begin{itemize}
      \item Concepto de suma no commutativa.
      \item Restricciones sobre el orden.
    \end{itemize}
\end{itemize}
\end{frame}

\begin{frame}[t]{GENERALIZED\_SUM}
\begin{itemize}
  \item \cppid{GENERALIZED\_SUM(op, a1, \ldots, aN)}:
    \begin{itemize}
      \item \cppid{op(GENERALIZED\_SUM(op,b1,\ldots, bK), GENERALIZED\_SUM(op,bL, \ldots, bN))}.
      \item \cppid{b1, \ldots, bK,bL, \ldots, bN} es una permutación de \cppid{a1, ldots, aN}.
    \end{itemize}
  \vfill
  \item \cppid{GENERALIZED\_NONCOMMUTATIVE\_SUM(op, a1, \ldots, aN)}:
    \begin{itemize}
      \item \cppid{op(GENERALIZED\_SUM(op,a1,\ldots, aK), GENERALIZED\_SUM(op,aL, \ldots, bN))}.
    \end{itemize}
\end{itemize}
\end{frame}

\begin{frame}[t,fragile]{reduce}
\begin{lstlisting}g
template <class InputIterator, class T, class BinaryOperation>
T reduce(InputIterator first, InputIterator last, T init, BinaryOperation op);
\end{lstlisting}
\begin{itemize}
  \item \cppid{GENERALIZED\_SUM(op, init, *first, \ldots )}
  \item \cppid{op} no puede invalidar iteradores en el rango tratado.
  \item \cppid{op} no puede modificar valores almacenados en el rango.
  \item \cppid{reduce} puede no ser determinista cuando \cppid{op} es no asociativo o no conmutativo.
\end{itemize}
\begin{block}{Ejemplo}
\begin{lstlisting}g
reduce(begin(v), end(v), 0, [](int a, int b) { 
  return a+b; });
\end{lstlisting}
\end{block}
\end{frame}

\begin{frame}[t,fragile]{Variantes de reduce}
\begin{lstlisting}g
template <class InputIterator, class T, class BinaryOperation>
T reduce(InputIterator first, InputIterator last, T init);
\end{lstlisting}
\begin{itemize}
  \item Utiliza como operación por defecto la suma.
  \item \cppid{reduce(first,last,init,plus<>())}.
\end{itemize}
\pause
\begin{lstlisting}g
template <class InputIterator, class T, class BinaryOperation>
T reduce(InputIterator first, InputIterator last);
\end{lstlisting}
\begin{itemize}
  \item Utiliza como operación por defecto la suma.
  \item Utiliza como valor por defecto el valor por defecto para el tipo usado para valores.
  \item \cppid{reduce(first,last,x,plus<>())}.
    \begin{itemize}
      \item \cppid{x} es el valor por defecto para el tipo de valor.
      \item \cppid{decltype(*first) x\{\};}
    \end{itemize}
\end{itemize}
\end{frame}

\begin{frame}[t,fragile]{Transform reduce}
\begin{lstlisting}g
template <class InputIterator, class T, class UnaryOp, class BinaryOp>
T transform_reduce(InputIterator first, InputIterator last, 
                   UnaryOp unary_op, T init, BinaryOp binary_op);
\end{lstlisting}
\begin{itemize}
  \item Realiza una reducción con \cppid{binary\_op} con el resultado de aplicar \cppid{unary\_op} a los elementos.
    \begin{itemize}
      \item \cppid{unary\_op} no se aplica a \cppid{init}.
    \end{itemize}
  \item Reducciones realizadas con \cppid{GENERALIZED\_SUM} sobre \cppid{binary\_op}.
\end{itemize}
\begin{block}{Ejemplo}
\begin{lstlisting}g
auto suma_cuadrados = transform_reduce(begin(v), end(v), 
  [](double x) { return x * x; },
  0,
  [](double x, double y) { return x + y; }
);
\end{lstlisting}
\end{block}
\end{frame}

\begin{frame}[t,fragile]{scan}
\begin{itemize}
  \item Dos variantes:
    \begin{itemize}
      \item \cppid{inclusive\_scan}: Realiza \emph{scan} hasta \emph{i-ésimo} elemento.
      \item \cppid{exclusive\_scan}: Realiza \emph{scan} hasta justo antes del \emph{i-ésimo} elemento.
    \end{itemize}
  \item En ambos casos:
    \begin{itemize}
      \item Las operaciones se realizan en términos de \cppid{GENERALIZED\_NONCOMMUTATIVE\_SUM}.
      \item La operación no puede invalidar ni iteradores ni modificar elementos.
    \end{itemize}
\end{itemize}
\begin{block}{Ejemplo}
\begin{lstlisting}g
auto fin = exclusive_scan(begin(v), end(v), begin(w));
\end{lstlisting}
\end{block}
\end{frame}

\begin{frame}[t,fragile]{exclusive scan}
\begin{lstlisting}g
template <class InputIterator, class OutputIterator, class T, 
          class BinaryOp>
OutputIterator exclusive_scan(InputItertor first, InputIterator last, 
                              OutputIterator out,
                              T init, BinaryOp op);
\end{lstlisting}
\begin{itemize}
  \item Para cada \cppid{i} en el rango \cppid{[out, out + (last-first))}:
    \begin{itemize}
      \item Le asigna el resultado de 
        \cppid{GENERALIZED\_NONCOMMUTATIVE\_SUM(op, init, *first, \ldots, *(first+(i-out)-1))}.
    \end{itemize}
  \item Devuelve el fin del rango resultado.
  \vfill
  \item Si se omite \cppid{op} equivale a:
    \begin{itemize}
      \item \cppid{exclusive\_scan(first, last, out, init, plus<>())}
    \end{itemize}
  \item No se puede omitir \cppid{init}.
\end{itemize}
\end{frame}

\begin{frame}[t,fragile]{inclusive scan}
\begin{lstlisting}g
template <class InputIterator, class OutputIterator, class T, 
          class BinaryOp>
OutputIterator inclusive_scan(InputItertor first, InputIterator last, 
                              OutputIterator out,
                              T init, BinaryOp op);
\end{lstlisting}
\begin{itemize}
  \item Para cada \cppid{i} en el rango \cppid{[out, out + (last-first))}:
    \begin{itemize}
      \item Le asigna el resultado de 
        \cppid{GENERALIZED\_NONCOMMUTATIVE\_SUM(op, init, *first, \ldots, *(first+(i-out)))}.
    \end{itemize}
  \item Devuelve el fin del rango resultado.
  \vfill
  \item Si se omite \cppid{op} equivale a:
    \begin{itemize}
      \item \cppid{exclusive\_scan(first, last, out, init, plus<>())}
    \end{itemize}
  \item Si se omite \cppid{init} no se incorpora a la suma generalizada.
\end{itemize}
\end{frame}

\begin{frame}[t,fragile]{Ejemplo}
\begin{block}{Calculando la CDF de un histograma}
\begin{lstlisting}g
vector<int> histogram = compute_histogram();
vector<int> cdf(histogram.size(), 0);
inclusive_scan(par, begin(histogram), end(histogram),
  begin(cdf));
\end{lstlisting}
\end{block}
\end{frame}

\begin{frame}[t,fragile]{transform scan}
\begin{itemize}
  \item Una versión para \cppid{transform\_exclusive\_scan}
\end{itemize}
\begin{lstlisting}g
template <class InputIterator, class OutputIterator, 
          class UnaryOp, class T, class BinaryOp>
OutputIterator transform_exclusive_scan(
  InputItertor first, InputIterator last, OutputIterator out, 
  UnaryOp unary_op, T init, BinaryOp binary_op);
\end{lstlisting}
\begin{itemize}
  \item Dos versiones para \cppid{transform\_inclusive\_scan}
\end{itemize}
\begin{lstlisting}g
template <class InputIterator, class OutputIterator, 
          class UnaryOp, class BinaryOp>
OutputIterator transform_inclusive_scan(
  InputItertor first, InputIterator last, OutputIterator out, 
  UnaryOp unary_op, BinaryOp binary_op);
template <class InputIterator, class OutputIterator, 
          class UnaryOp, class BinaryOp, class T>
OutputIterator transform_exclusive_scan(
  InputItertor first, InputIterator last, OutputIterator out, 
  UnaryOp unary_op, BinaryOp binary_op, T init);
\end{lstlisting}
\end{frame}

\begin{frame}[t,fragile]{Ejemplo}
\begin{block}{Calculando la CDF de un histograma}
\begin{lstlisting}g
vector<int> histogram = compute_histogram();
vector<int> cdf(histogram.size(), 0);
transform_inclusive_scan(par, begin(histogram), end(histogram),
  begin(cdf)
  [](int x) {
    if (x<0) return 0;
    if (x>255) return 255;
    return x;
  }
  [](int x, int y) { return x+y; });
\end{lstlisting}
\end{block}
\end{frame}
